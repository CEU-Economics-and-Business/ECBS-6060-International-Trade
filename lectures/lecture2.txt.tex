\documentclass[compress,mathserif,aspectratio=169]{beamer}
\usepackage[latin2]{inputenc}
%\usepackage[absolute]{textpos}
%\documentclass[handout,compress,mathserif]{beamer}
%\setbeameroption{show notes}

% This file is a solution template for:

% - Talk at a conference/colloquium.
% - Talk length is about 20min.
% - Style is ornate.



% Copyright 2004 by Till Tantau <tantau@users.sourceforge.net>.
%
% In principle, this file can be redistributed and/or modified under
% the terms of the GNU Public License, version 2.
%
% However, this file is supposed to be a template to be modified
% for your own needs. For this reason, if you use this file as a
% template and not specifically distribute it as part of a another
% package/program, I grant the extra permission to freely copy and
% modify this file as you see fit and even to delete this copyright
% notice.


\mode<presentation>
{
%  \usetheme{pittsburgh}
  % or ...

  \setbeamercovered{invisible}
  % or whatever (possibly just delete it)
}


\usepackage[USenglish]{babel}
%\usepackage[latin1]{inputenc}
%\usepackage[T1]{fontenc}
\usepackage{ifthen,array}



\pretolerance5000 \hyphenpenalty9999
%\setlength{\TPHorizModule}{0.5cm} \setlength{\TPVertModule}{0.5cm}
%\textblockorigin{20mm}{20mm} % start everything near the top-left corner

\newcounter{ora}
\newcounter{perc}
\newcounter{kezdoora}
\newcounter{kezdoperc}
\newcounter{percek}
\setcounter{percek}{15}
\setcounter{kezdoora}{4} % for 1.35pm as the starting time

\providecommand{\leadingzero}[1]{\ifthenelse{\value{#1}<10}{0\arabic{#1}}{\arabic{#1}}}
\providecommand{\oradisplay}[1]{\ifthenelse{\value{#1}<60}{\arabic{kezdoora}:\leadingzero{#1}}{\setcounter{perc}{\value{#1}}\addtocounter{perc}{-60}\setcounter{ora}{\value{kezdoora}}\addtocounter{ora}{1}\arabic{ora}:\leadingzero{perc}}}

\providecommand{\notes}[1]{{\tiny\textbf{Note:} #1}}
%%%%%%%%%%%%%%%%%%%%%%%%%%%%%%%%%%%%%%%%%%%%%%%%
%% Hasznos matek makrok
%%%%%%%%%%%%%%%%%%%%%%%%%%%%%%%%%%%%%%%%%%%%%%%%

\newcommand{\QED}{{}\hfill$\Box$}
\newcommand{\intl}[4]{\int_{#1}^{#2} \! {#3} \, \mathrm d{#4}}
\newcommand{\period}{\text{.}} % Ez azert kell, mert a matek . mashogy nez ki, mint a szovege.
\newcommand{\comma}{\text{,}}  % Ez azert kell, mert a matek , mashogy nez ki, mint a szovege.
\newcommand{\dist}{\,\mathop{\operatorname{\sim\,}}\limits}
\newcommand{\D}{\,\mathop{\operatorname{d}}\!}
%\newcommand{\E}{\mathop{\operatorname{E}}\nolimits}
\newcommand{\Lag}{\mathop{\operatorname{L}}}
\newcommand{\plim}{\mathop{\operatorname{plim}}\limits_{T\to\infty}\,}
\newcommand{\CES}[3]{\mathop{\operatorname{CES}}\left(\left\{#1\right\},\left\{#2\right\},#3\right)}
\newcommand{\cestwo}[5]{\left[#1^\frac1{#5}\,#2^\frac{#5-1}{#5}+#3^\frac1{#5}\,#4^\frac{#5-1}{#5}\right]^\frac{#5}{#5-1}}
\newcommand{\cesmore}[4]{\left[\sum_{#3}#1_{#3}^\frac1{#4}\,{#2}_{#3}^\frac{#4-1}{#4}\right]^\frac{#4}{#4-1}}
\newcommand{\cesPtwo}[5]{\left[#1\,#2^{1-#5}+#3\,#4^{1-#5}\right]^\frac{1}{1-#5}}
\newcommand{\cesPmore}[4]{\left[\sum_{#3}#1_{#3}\,#2_{#3}^{1-#4}\right]^\frac{1}{1-#4}}
\newcommand{\diff}[2]{\frac{\D #1}{\D #2}}
\newcommand{\pdiff}[2]{\frac{\partial #1}{\partial #2}}
\newcommand{\convex}[2]{\lambda #1 + (1-\lambda)#2}
\newcommand{\ABS}[1]{\left| #1 \right|}
\newcommand{\suchthat}{:\hskip1em}
\newcommand{\dispfrac}[2]{\frac{\displaystyle #1}{\displaystyle #2}} % Emeletes tortekhez hasznos.

\newcommand{\diag}{\mathop{\mathrm{diag\mathstrut}}}
\newcommand{\tr}{\mathop{\mathrm{tr\mathstrut}}}
\newcommand{\E}{\mathop{\mathrm{E\mathstrut}}}
\newcommand{\Var}{\mathop{\mathrm{Var\mathstrut}}\nolimits}
\newcommand{\Cov}{\mathop{\mathrm{Cov\mathstrut}}}
\newcommand{\sgn}{\mathop{\operatorname{sgn\mathstrut}}}

\newcommand{\covmat}{\mathbf\Sigma}
\newcommand{\ones}{\mathbf 1}
\newcommand{\zeros}{\mathbf 0}
\newcommand{\BAR}[1]{\overline{#1}}

\renewcommand{\time}[1]{\addtocounter{percek}{#1}}

\newlength{\tempsep}

\newenvironment{subeqs}{\setlength{\tempsep}{\arraycolsep}
\setlength{\arraycolsep}{0.13889em} % Ez azert kell, hogy ne hagyjon tul sok helyet az = korul.
\begin{subequations}\begin{eqnarray}}
{\end{eqnarray}\end{subequations}
\setlength{\arraycolsep}{\tempsep}}

\newenvironment{tapad}{\setlength{\tempsep}{\arraycolsep}
\setlength{\arraycolsep}{0.13889em}} % Ez azert kell, hogy ne hagyjon tul sok helyet az = korul.
{\setlength{\arraycolsep}{\tempsep}}

\newenvironment{eqnarr}{\setlength{\tempsep}{\arraycolsep}
\setlength{\arraycolsep}{0.13889em} % Ez azert kell, hogy ne hagyjon tul sok helyet az = korul.
\begin{eqnarray}}
{\end{eqnarray} \setlength{\arraycolsep}{\tempsep}}

\newenvironment{eqnarr*}{\setlength{\tempsep}{\arraycolsep}
\setlength{\arraycolsep}{0.13889em} % Ez azert kell, hogy ne hagyjon tul sok helyet az = korul.
\begin{eqnarray*}}
{\end{eqnarray*} \setlength{\arraycolsep}{\tempsep}}


%\usepackage[active]{srcltx} % SRC Specials: DVI [Inverse] Search
% Fuzz --- -------------------------------------------------------
\hfuzz5pt % Don't bother to report over-full boxes < 5pt
\vfuzz5pt % Don't bother to report over-full boxes < 5pt
% THEOREMS -------------------------------------------------------
% MATH -----------------------------------------------------------
\newcommand{\norm}[1]{\left\Vert#1\right\Vert}
\newcommand{\abs}[1]{\left\vert#1\right\vert}
\newcommand{\set}[1]{\left\{#1\right\}}
\newcommand{\Real}{\mathbb R}
\newcommand{\eps}{\varepsilon}
\newcommand{\To}{\longrightarrow}
\newcommand{\BX}{\mathbf{B}(X)}
\newcommand{\A}{\mathcal{A}}




\newcommand{\directory}{figures}
\newcommand*{\newtitle}{\egroup\begin{frame}\frametitle}

\newcommand{\fullpagefigure}[2]{\begin{frame}\frametitle{\hyperlink{#1back}{#2}}\hypertarget{#1}{{\begin{center}\includegraphics[height=0.9\textheight]{\directory/#1}\end{center}}}\end{frame}}
\newcommand{\widefigure}[2]{\begin{frame}\frametitle{\hyperlink{#1back}{#2}}\hypertarget{#1}{{\begin{center}\includegraphics[width=\linewidth]{\directory/#1}\end{center}}}\end{frame}}
\newcommand{\longfigure}[2]{\begin{frame}\frametitle{\hyperlink{#1back}{#2}}\hypertarget{#1}{{\begin{center}\includegraphics[height=0.8\textheight]{\directory/#1}\end{center}}}\end{frame}}
%\newcommand{\fullpagefigure}[2]{\begin{frame}\frametitle{\hyperlink{#1back}{#2}}\hypertarget{#1}{{\begin{centering}$#1$\end{centering}}}\end{frame}}
\newcommand{\answer}[1]{\begin{itemize}\item #1\end{itemize}}


\newcommand{\jumpto}[2]{\hypertarget{#1back}{\hyperlink{#1}{#2}}}
\newcommand{\backto}[2]{\hypertarget{#1}{\hyperlink{#1back}{#2}}}


\title{ECBS 6060: International Trade\\
Winter 2020}

\author{Mikl\'os Koren\\
korenm@ceu.edu}
% - Give the names in the same order as the appear in the paper.
% - Use the \inst{?} command only if the authors have different
%   affiliation.


\date % (optional, should be abbreviation of conference name)
{}
% - Either use conference name or its abbreviation.
% - Not really informative to the audience, more for people (including
%   yourself) who are reading the slides online

%\subject{Theoretical Computer Science}
% This is only inserted into the PDF information catalog. Can be left
% out.



% If you have a file called "university-logo-filename.xxx", where xxx
% is a graphic format that can be processed by latex or pdflatex,
% resp., then you can add a logo as follows:

\pgfdeclareimage[height=0.5cm]{university-logo}{frblogo}
%\logo{\pgfuseimage{university-logo}}



% Delete this, if you do not want the table of contents to pop up at
% the beginning of each subsection:
\AtBeginSection[]
{
  \begin{frame}[plain]
    \frametitle{\color{red}\insertsection}
    \addtocounter{framenumber}{-1}
    %\tableofcontents[currentsection,currentsubsection]
  \end{frame}
}


% If you wish to uncover everything in a step-wise fashion, uncomment
% the following command:

%\beamerdefaultoverlayspecification{<+->}

\setbeamertemplate{navigation symbols}{}
\setbeamertemplate{footline}{{}\hfill\insertframenumber}

\begin{document}

\begin{frame}[plain]
  \titlepage
    \addtocounter{framenumber}{-1}
\end{frame}




\section{Lecture 2: Patterns of trade}\hypertarget{Lecture 2: Patterns of trade}{}


\begin{frame}\frametitle{Setup}\hypertarget{Setup}{}
The country has 
\begin{itemize}
\item endowments (vector) $\mathbf v$

\item production $\mathbf x$

\item consumption $\mathbf c$

\item prices $p$


\end{itemize}

\end{frame}



\begin{frame}\frametitle{Technology and tastes}\hypertarget{Technology and tastes}{}
\begin{itemize}
\item Technology is represented by the revenue function
\[
r(p,\mathbf v) = \max_{\mathbf x} p\mathbf x: (\mathbf x,\mathbf v) \text{ feasible}.
\]

\item Tastes are represented by the expenditure function
\[
e(p,u) = \min_{\mathbf c} p\mathbf c: u(\mathbf c)=u.
\]


\end{itemize}
\end{frame}



\begin{frame}\frametitle{Supply and demand}\hypertarget{Supply and demand}{}
\begin{itemize}
\item Supply of goods (production)
\[
\mathbf x = r_p(p,\mathbf v).
\]

\item Demand for goods (consumption)
\[
\mathbf c = e_p(p,u).
\]


\end{itemize}
\end{frame}



\begin{frame}\frametitle{Autarky equilibrium}\hypertarget{Autarky equilibrium}{}
\begin{itemize}
\item \emph{Autarky}: the economy is closed, markets have to clear \emph{within} the country.

\item In equilibrium, each product market clears,
\[
r_p(p,v) = e_p(p,u).
\]

\item Expenditure equals revenue
\[
r(p,v) = e(p,u).
\]
\pause



\item What about factor markets?

\item What about profit maximization?

\item What about utility maximization?


\end{itemize}
\end{frame}



\begin{frame}\frametitle{Trading equilibrium}\hypertarget{Trading equilibrium}{}
\begin{itemize}
\item We start with the small open economy.

\item The world price is $p^w$ (arbitrary).

\item At this price, the world buys and sells any amount.

\item Net import of goods: 
\[
\mathbf m(p^w) = \mathbf c(p^w) - \mathbf x(p^w).
\]


\end{itemize}
\end{frame}



\begin{frame}\frametitle{Equilibrium conditions}\hypertarget{Equilibrium conditions}{}
\begin{itemize}
\item Balanced trade:
\[
0 = p^w \mathbf m = p^w(\mathbf c^w-\mathbf x^w) = 
p^w[e_p(p^w,u^w) - r_p(p^w,\mathbf v)].
\]

\item Use Euler's theorem:
\[
r(p,\mathbf v) = e(p,u).
\]




\end{itemize}
\end{frame}



\begin{frame}\frametitle{Equilibrium conditions}\hypertarget{Equilibrium conditions}{}
\begin{itemize}
\item Goods market are now global, so local markets do not have to clear.
\[
\mathbf m(p^w) = e_p(p,u) - r_p(p,\mathbf v).
\]

\item What about factor markets?




\end{itemize}
\end{frame}



\begin{frame}\frametitle{Gains from trade}\hypertarget{Gains from trade}{}
\begin{itemize}
\item Let $p^a$ denote the vector of autarky prices.
\begin{align*}
 e(p^w,u^a) & \le p^w\mathbf c^a\\
    &= p^w\mathbf x^a\\
    &\le r(p^w,\mathbf v)\\
    & = e(p^w,u^w)
\end{align*}
\pause


\begin{enumerate}\setcounter{enumi}{0}
\item Definition of the expenditure function.

\item Autarky equilibrium.

\item Definition of the revenue function.

\item Open-economy equilibrium.


\end{enumerate}

\end{itemize}
\end{frame}



\begin{frame}\frametitle{Gains from trade}\hypertarget{Gains from trade}{}
\begin{itemize}
\item Since $e(p,u)$ is increasing in $u$,
\[
 u^w\ge u^a.
\]

\item Welfare is higher under trade than under autarky.

\item Note that this holds for any $p^w$.


\end{itemize}
\end{frame}



\begin{frame}\frametitle{Graphical illustration with PPS}\hypertarget{Graphical illustration with PPS}{}


\end{frame}



\begin{frame}\frametitle{Discussion}\hypertarget{Discussion}{}
\begin{itemize}
\item Moving from autarky to free trade always improves \emph{aggregate welfare}.

\item What we assumed:
\begin{enumerate}\setcounter{enumi}{0}
\item representative consumer/producer

\item constant returns to scale technologies

\item perfect competition

\item no externalities / market failures


\end{enumerate}

\end{itemize}
\end{frame}



\begin{frame}\frametitle{The equivalence of trade and technology}\hypertarget{The equivalence of trade and technology}{}
\begin{itemize}
\item Trade is a ``technology" to transform export goods into import goods.

\item As long as technology use is voluntary, having access to it is welfare improving.

\item There may be important distributional consequences (to be discussed later).
\begin{itemize}
\item Attitudes toward trade should be similar to attitudes toward technical progress.


\end{itemize}

\end{itemize}
\end{frame}



\begin{frame}\frametitle{2-country case}\hypertarget{2-country case}{}
\begin{itemize}
\item Because the country gains for any $p^w$, when two countries open up to trade, \emph{both gain}.










\end{itemize}
\end{frame}







\section{Patterns of trade}\hypertarget{Patterns of trade}{}
\begin{frame}\frametitle{Patterns of trade}\hypertarget{Patterns of trade}{}
\begin{itemize}
\item How much can we say about the patterns of trade without talking about technologies, endowments, and tastes?

\item Quite a lot. These are all summarized in the \emph{autarky equilibrium price}.

\item Deardorff (1980, JPE) derived the law of comparative advantage.


\end{itemize}
\end{frame}



\begin{frame}\frametitle{The law of comparative advantage}\hypertarget{The law of comparative advantage}{}
\begin{itemize}
\item Balanced trade:
\[
p^w \mathbf m = 0.
\]

\item Autarky consumption is affordable at world prices,
\[
p^w\mathbf c^a \le p^w\mathbf c^w.
\]
Why?
\pause


\begin{enumerate}\setcounter{enumi}{0}
\item $p^w\mathbf x^a\le p^w\mathbf x^w$ by revenue maximization. ($\mathbf x^a$ is still feasible to produce, but brings less revenue.)

\item $p^w\mathbf x^a=p^w\mathbf c^a$ by market clearing.

\item $p^w\mathbf x^w=p^w\mathbf c^w$ by balanced trade.




\end{enumerate}

\end{itemize}
\end{frame}



\begin{frame}\frametitle{The law of comparative advantage}\hypertarget{The law of comparative advantage}{}
\begin{itemize}
\item Autarky consumption is affordable under free trade, yet not chosen.

\item By the \emph{weak axiom of revealed preference}:
\[
p^a\mathbf c^w > p^a\mathbf c^a.
\]

\item By revenue maximization,
\[
p^a\mathbf x^w \le p^a\mathbf x^a.
\]

\item Subtract the value of production, $p^a\mathbf x^w$ and $p^a \mathbf x^a$:
\[
p^a(\mathbf c^w-\mathbf x^w) = p^a\mathbf m > p^a(\mathbf c^a-\mathbf x^a)=0.
\]

\item \emph{The autarky cost of the net import vector is positive}.

\item Equivalently,
\[
(p^a-p^w)\mathbf m > 0.
\]


\end{itemize}
\end{frame}



\begin{frame}\frametitle{The 2-good case}\hypertarget{The 2-good case}{}
\begin{itemize}
\item For two goods, this means $m_i>0$ if and only if $p^a_i>p^w_i$.

\item The country exports the product which has a low autarky price relative to the world price.

\item The country imports the product which has a high autarky price relative to the world price.

\item (How does this relate to ``Buy cheap, sell dear?")

\item There is no such strong conclusion for the $n$-good case. We only have a correlation of prices and trade patterns.




\end{itemize}
\end{frame}



\begin{frame}\frametitle{The 2-country case}\hypertarget{The 2-country case}{}
\begin{itemize}
\item In a 2-country world, net imports of country 1 are net exports of country 2:
\[
\mathbf m = -\mathbf M.
\]

\item (We use uppercase letters for country 2.)

\item The law of CA holds in both countries
\begin{align*}
p^a\mathbf m &>0,\\
P^a\mathbf M &>0.
\end{align*}

\item Summing the two
\[
(p^a-P^a) \mathbf m >0.
\]


\end{itemize}
\end{frame}



\begin{frame}\frametitle{The 2$\times$2 case}\hypertarget{The 2$\times$2 case}{}
%% redo this
\begin{itemize}
\item Goods flow from the low autarky price country to the high autarky price country.
\[
(p^a-P^a) \mathbf m >0.
\]

\item Balanced trade
\[
p^w m = 0.
\]

\item With two goods, this implies
\[
\frac{p^a_1}{p^a_2} < \frac{p^w_1}{p^w_2} < \frac{P^a_1}{P^a_2}
\]
if good 1 is exported and the reverse if good 2..

\item We can narrow down the prices in trade equilibrium.

\item Again, no such strong conclusion for the $n$-good case.






\end{itemize}
\end{frame}







\section{Evidence on general trade theorems}\hypertarget{Evidence on general trade theorems}{}






\section{Trade and welfare}\hypertarget{Trade and welfare}{}
\begin{frame}\frametitle{Estimable equation}\hypertarget{Estimable equation}{}
\begin{itemize}
\item Let $y_i=Y_i/L_i$ denote GDP \emph{per capita} of country $i$.

\item $T_i=(X_i+M_i)/Y_i$ measures the trade openness of country $i$.

\item One way of capturing the benefits of trade is to say that
\begin{itemize}
\item per capita income is increasing in openness.
\pause


\end{itemize}

\[
y_i = \beta_0 + \beta_1 T_i + u_i
\]


\end{itemize}
\end{frame}



\begin{frame}\frametitle{Identification problem}\hypertarget{Identification problem}{}
\begin{itemize}
\item $\beta_1$ should measure the effect of a change in $T_i$ \emph{holding other factors fixed} ($u_i$)

\item Countries with high $T_i$ have different $u_i$.
\begin{itemize}
\item \alert{Reverse causality}: richer countries trade more.

\item \alert{Omitted variables}: institutions, macro policies may affect both $y_i$ and $T_i$.
\end{itemize}

\item Both lead to an \alert{endogeneity bias}.
\begin{itemize}
\item Coefficient estimated by OLS does not capture \emph{causal} effect.


\end{itemize}

\end{itemize}
\end{frame}



\begin{frame}\frametitle{Natural experiments}\hypertarget{Natural experiments}{}
\begin{itemize}
\item These are events creating variation in openness \emph{uncorrelated with} income.

\item Two key assumptions:
\begin{enumerate}\setcounter{enumi}{0}
\item event (or other variable) correlated with trade

\item holding trade fixed, uncorrelated with income




\end{enumerate}

\end{itemize}
\end{frame}



\begin{frame}\frametitle{Four natural experiments}\hypertarget{Four natural experiments}{}
\begin{enumerate}\setcounter{enumi}{0}
\item Japan 1851

\item US 1808

\item Suez Canal 1967

\item Air freight 1950--1990




\end{enumerate}
\end{frame}







\section{Law of comparative advantage}\hypertarget{Law of comparative advantage}{}
\begin{frame}\frametitle{Empirical tests}\hypertarget{Empirical tests}{}
\begin{itemize}
\item It is difficult to test these predictions directly, because autarky prices are a counterfactual construct.

\item We rarely see countries both in trade and in autarky with the same technology, endowments and preferences.

\item Bernhofen and Brown (2004, JPE): Japan's opening in the 19th century.

\item Irwin (2005, RIE): self-imposed trade embargo of the U.S. in 1808.


\end{itemize}
\end{frame}



\begin{frame}\frametitle{The natural experiment of Japan}\hypertarget{The natural experiment of Japan}{}
\begin{itemize}
\item Japan cut itself off from all trade between 1639 and 1853.

\item (Except for a small Dutch trading post off the Nagasaki harbor.)

\item At the pressure of Western powers, it opened rapidly between 1853 and 1858.


\end{itemize}
\end{frame}



\begin{frame}\frametitle{Measurement}\hypertarget{Measurement}{}
\begin{itemize}
\item Autarky: 1851--53 (so that production possibilities and tastes are close to trade period)

\item Trade: 1868--75.

\item Prices are from merchant's books.


\end{itemize}
\end{frame}




\longfigure{japan-trade}{Export and import commodities of Japan}
\longfigure{autarky-prices}{Change in prices and net export}
\widefigure{lawofca}{Autarky cost of the net export vector}


\begin{frame}\frametitle{Results}\hypertarget{Results}{}
\begin{itemize}
\item Autarky prices of export goods were generally lower than trade prices.

\item Autarky prices of import goods were generally higher than trade prices.

\item The autarky cost of the net import vector was negative.

\item Confirming the law of comparative advantage.

\item Bernhofen and Brown (2005, AER): autarky may have cost up to 8--9\% of Japan's GDP


\end{itemize}
\end{frame}



\begin{frame}\frametitle{The natural experiment in the U.S.}\hypertarget{The natural experiment in the U.S.}{}
\begin{itemize}
\item In 1807, U.S. Congress enacted an embargo on maritime trade.
\begin{enumerate}\setcounter{enumi}{0}
\item To protect American vessels and crew from British (and French) harassment.

\item To put economic costs on Britain.
\end{enumerate}

\item The embargo was widely observed and almost "sealed off" the U.S. from the ROW.

\item Domestic prices of export and import goods have changed substantially.


\end{itemize}
\end{frame}




\widefigure{tonnage}{Tonnage of American ships entering Britain}
\widefigure{us-exports}{Domestic wholesale price of export commodities}
\widefigure{us-imports}{Price index of import commodities}


\longfigure{suez-1st-stage}{The Suez Canal closure reduced trade between country pairs (Feyrer 2009)}
\widefigure{suez-2nd-stage}{Thereby leading to lower GDP per capita (Feyrer 2009)}


\begin{frame}\frametitle{Results}\hypertarget{Results}{}
\begin{itemize}
\item Exports became cheaper, imports became more expensive.

\item Consistently with the law of comparative advantage.








\end{itemize}
\end{frame}







\end{document}